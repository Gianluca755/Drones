\documentclass[a4paper, oneside]{memoir}
\usepackage{geometry}	% layout margin
\usepackage{graphicx}
\graphicspath{ {./images/} }

\title{How to write a Report\\ for the Project of Distributed Systems}

\author{Prof. Marino Miculan\\ DMIF, University of Udine, Italy}

\date{Version 0.2, \today}

\begin{document}


%\begin{titlingpage}
\maketitle
%\begin{abstract}
%This document present a distributed system of drones for the delivery of packages. The goal is to a solution that handles orders from clients and provide a reliable fleet of drones, for the delivery.
%\end{abstract}
%\end{titlingpage}

\chapter{Introduction}\label{ch:intro}

The problem to solve is: the handle of delivery orders and management of a dynamic fleet of drones.\\
The space is modelled by a 2D grid, so the diagonal movements of drones are calculated with some approximation. There aren't forbidden zones nor forced paths.\\
The drones have cellular connectivity and know a few other drones. They are equipped with GPS.\\
There are some recharging and repairing stations on the map.\\
There is a server that centralize the orders of the clients and act as a reliable message broker (middle-man).\\
There is a server which provide some management of drones, and make sure the order is carried out.\\
There is an other server for the registration of the drones; when a new drone wants to join the fleet this server will give the new drone an ID and a list of other drones to connect to and the locations of recharging stations and warehouses.

\begin{figure}[h!]
	\centering
	\includegraphics[width=\linewidth]{Overview}
\end{figure}

A client can send a request to the broker server for the delivery of a package from point A to point B, a done will be selected based on it's position relative to point A, it's  battery and it's maximum carry weight, the selected drone will carry out the delivery of the package.
If a drone has low battery it will enter reserve mode, it will locate the nearest charging station and it will move to it, to recharge its battery, the charging stations are fixed and their location is known to all drones.\\
A drone can crash at any point in time, when this happens another drone will be selected to transport the broken drone to the nearest charging station, if the broken drone was satisfying a request a second drone will be selected to complete the task.

\begin{itemize}
\item The distributed system features (and the transparencies) and algorithms you intend to implement.

% \item Your plan for testing the system.

The test cases that will be using must be sufficient to test all the critical situations that the system can encounter, those situations are:
\begin{itemize}
\item the case in which a drone crashes while it's idle;
\item the case in which a drone crashes after it has been selected to move a package but it has not picked up the package yet:
\item the case in which a drone crashes after it has picked up the package and it's delivering it to point B;
\item the case in which a drone crashes after it has sent it's featurese(battery, position and carry weight) to the network for election;
\item the case in which a drone has to go to a charging station;
\item the case in which no drone is available to deliver a package;
\item the case in which one or more drones lie about their position;
\item the case in which a new drone joins the network;
\item the case in which a drone crashes after it has sent a join request to the server.
\end{itemize}




\item A schedule for how you plan to carry our your design and implementation
\end{itemize}



\chapter{Analysis}\label{ch:analysis}

%In this chapter, we describe in detail functional and non-functional requirements of a solution for the problem.

\section{Functional requirements}
%Which functions must be offered to users / other programs?  Which are the input data and the output data? Which is the expected effect?
The following are the functional requirements required by the project:
\begin{itemize}
\item a client can request the delivery of a package form point A to point B;
\item a client recieves an ack message if their request has been recieved;
\item a drone can crash at any point;
\item if a drone crashes another one will complete it's task;
\item a broken drone will eventually be moved to a charging station for repairs;
\item there must be a process for new drones to join the network.
\end{itemize}

\section{Non functional requirements}
%Everything about mode and transparencies: availability, mobility, security, fault tolerance, etc.
The following are the non-functional requirements required by the project:
\begin{itemize}
\item it's important for following transparencies to be present:
	\begin{itemize}
	\item concurrency: only one drone will be selected to carry out the delivery of the package;
	\item reconfiguration: the entrance of a new drone in the network must have a low overhead;
	\item failure: if a drone crashes the delivery must be completed smothly for the client;
	\item performance and scaling: the system must be able to scale smoothly when adding more drones .
	\end{itemize}
\end{itemize}
%Are there execution time bounds? Minimum data rates?

%If requested, specific platforms/languages/middlewares requirements for the implementation can be decided here. (E.g.: if the project is on a SOA, we may request that functions are offered via SOAP or RESTful services).



\chapter{Project}

This chapter is devoted to the description of the general architectures, and specific algorithms.

\section{Logical architecture}
Describe the components of your systems: modules/objects/components/services.
For each component, describe the functionalities it implements, and by who is used.

\section{Protocols and algorithms}
Communication between components.  UML sequence diagrams go here.

Also, put here a detailed description of distributed algorithms used to solve specific problems of the project.

\section{Physical architecture and deployment}
Which nodes and platforms involved, and where each component is deployed.

\section{Development plan}
Since it is difficult to predict just how hard implementing a new system will be, you should formulate as a set of ``tiers,'' where the basic tier is something you’re sure you can complete, and the additional tiers add more features, at both the application and the system level.

\chapter{Implementation}

Details about the implementation: every choice about platforms, languages, software/hardware, middlewares, which has not been decided in the requirements.


Important choices about implementation should be described here; e.g., peculiar data structures.


\chapter{Validation}

Check if requirements from Chapter~\ref{ch:analysis} have been fulfilled.
Quantitative tests (simulations) and screenshots of the interfaces are put here.


\chapter{Conclusions}

What has been done with respect to what has been promised in Chapters~\ref{ch:intro} and \ref{ch:analysis}, and what is left out.

\appendix

\chapter{Appendix}

In the Appendix you can put code snippets, snapshots, installation instructions, etc.


\chapter*{Evaluation}
Your system will be judged mainly on how it operates as a distributed system. The primary evaluation will be according to whether your system has the following attributes:
\begin{itemize}
\item  It should be an interesting distributed system, making use of some of the algorithms we have covered in class for distributed synchronization, replication, fault tolerance and recovery, security, etc.
\item The software should be well designed and well implemented, in terms of the overall architecture and the detailed realization.
\item You should devise and apply systematic testing procedures, at both the unit and systems levels.
\item The system should operate reliably and with good performance, even in the face of failures.
\end{itemize}
Important, but secondary considerations include:
\begin{itemize}
\item Time taken to do the project (the sooner the better, but do not miss details in order to end sooner)
\item  How nice is the application's appearance: does it have a nice interface or a compelling visual display?
\end{itemize}

\end{document}